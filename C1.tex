\documentclass{ctexart}

\usepackage{amsmath}
\usepackage{amsfonts}
\usepackage{amssymb}
\usepackage{amsthm}
\usepackage{thmtools}
\usepackage{nameref}
\usepackage{hyperref}
\usepackage{xurl}

\theoremstyle{definition}
\newtheorem{definition}{定义}[section]
\newtheorem{lemma}{引理}[section]
\newtheorem{theorem}{定理}[section]
\newtheorem{proposition}{命题}[section]
\newtheorem{axiom}{公理}[section]

\title{OBSERVE Chapter 1}
\author{喵喵}
\date{2020.9}

\begin{document}
\maketitle

\tableofcontents

\section{关于“无穷集合”的定义}

\subsection{定义与引理}

方便起见,我们称 $0 = \emptyset$。

\begin{definition}[后继]
对于集合 $x$,称集合 x 的\textbf{后继} $succ(x) = x \cup \{ x \}$
\end{definition}

\begin{lemma}[后继的存在性]
  $\forall x: \exists y: y = succ(x)$
\end{lemma}

\begin{proof}
  根据 Axiom of pairing + Axiom schema of specification,$\{ x \}$ 一定存在。

  然后用 Axiom of pairing + Axiom schema of specification + Axiom of union,可以构造出 $succ(x)$
\end{proof}

\begin{definition}[自然数]\label{nn}
  称 $x$ 是自然数(Natural number) $\iff x = 0 \lor \exists y: y \text{是 \nameref{nn}} \land x = succ(y)$
\end{definition}

我们可以据此定义自然数之间的加法。根据 Axiom of regularity, 不存在自然数 x 和 y,使得 $y \neq 0 \land x + y = x$。这样根据后继定义的自然数不会打环。这样我们对于所有的自然数(现在还不一定是个集合)有了一个全序。

{
  \small
  这里其实有个洞:在存在无穷集合之前,我们能不能使用归纳定义呢?或者说这里这个定义并不涉及无穷次逻辑,所以其实没关系?\footnote{See: \url{https://math.stackexchange.com/questions/490825/prove-the-principle-of-mathematical-induction-in-sf-zfc}}
}

\vspace{3em}

根据自然数,我们可以定义有限集合和无穷集合。

\begin{definition}[有限集合]\label{finite}
我们称集合 $A$ 是\textbf{有限集合} $\iff \exists x: x \text{是 \nameref{nn}} \land A \sim x$
\end{definition}

\begin{definition}[无穷集合]\label{infinite}
我们称集合 $A$ 是\textbf{无穷集合} 当且仅当 $A$ 不是有限集合。
\end{definition}

\vspace{3em}

\begin{definition}[归纳集]\label{inductive}
我们称集合 $I$ 是\textbf{归纳集}(Inductive set) $\iff 0 \in I \land (\forall x \in I: succ(x) \in I)$
\end{definition}

据此定义我们可以陈述无穷公理(Axiom of Infinity):

\begin{axiom}[无穷公理]\label{aoi}
  $\exists I: I \text{是 \nameref{inductive}}$
\end{axiom}

我们暂时\textbf{不接受}无穷公理,而是希望证明这一公理和以下两个命题等价:

\begin{proposition}\label{pp1}
  $\exists A: A \text{是 \nameref{infinite}}$
\end{proposition}

\begin{proposition}[Dedekind-infinite]\label{pp2}
  $\exists A: \exists B: B \subsetneqq A \land A \sim B$
\end{proposition}

下证 \nameref{aoi} $\iff \text{命题 \ref{pp1}} \iff \text{命题 \ref{pp2}}$。

\vspace{3em}

在证明的过程中我们接受 ZFC 公理系统中所有除了无穷公理以外的公理,包括选择公理。

\begin{lemma}\label{lemma:cap-inductive}
  如果 A, B 均为 \nameref{inductive},那么 $A \cap B$ 也是 \nameref{inductive}。
\end{lemma}

\begin{proof}
  根据定义,A 和 B 显然都包含所有的自然数,因此 $A \cap B$ 也包含所有的自然数。
\end{proof}

\begin{lemma}
  \nameref{aoi} $\iff$ 存在自然数集 $\mathbb{N}$。
\end{lemma}

\begin{proof}
  由于 $\mathbb{N}$ 显然是 \nameref{inductive},所以 ($\impliedby$) 是不言自明的。下证 ($\implies$)

  根据 \nameref{aoi},令 $I$ 是 \nameref{inductive}。

  一个简单的证明方法是,根据 \nameref{inductive} 的定义,$I$ 包含所有的 \nameref{nn}。直接应用谓词"是 \nameref{nn}" 于\textbf{分离公理模式}(Axiom schema of specification),可以得到 $\mathbb{N} \subseteqq I$ 的存在性。
\end{proof}

自 \url{http://web.mat.bham.ac.uk/R.W.Kaye/logic/infinity.html} 的证明更加复杂,但是感觉避开了一些奇怪的味道?

\begin{proof}
  令 $P = \{S \subseteq I: S \text{是 \nameref{inductive}}\}$。根据\textbf{幂集公理}(Axiom of power set),并应用谓词"是 \nameref{inductive}" 于\textbf{分离公理模式}(Axiom schema of specification),可以得到 $P$ 的存在性。

  令 $\omega = \bigcap P$。根据引理 \ref{lemma:cap-inductive},$\mathbb{N}$ 是 \nameref{inductive}。

  现在我们构造出了一个\textbf{最小}的\nameref{inductive}。同样地,由于 $\omega$ 一定包含所有的自然数,因此它只包含所有的自然数。因此 $\mathbb{N} = \omega$。
\end{proof}

\subsection{三者等价性证明}

先来个引理。

\begin{lemma}\label{nat-ord}
  任意自然数比自身的任意真子集的势都要大。正式地:对于任意自然数 x,令谓词 $Nat(x)$ 表示“$x$ 是自然数”:

  $$
  \forall y: y \subsetneqq x \rightarrow (\exists z: Nat(z) \land z < x \land y \sim z)
  $$
\end{lemma}

\begin{proof}
  归纳

  初始条件:$0 = \emptyset$ 不存在真子集,因此这一命题对 0 显然成立。

  归纳假设:如果自然数 $x$ 满足上述条件,下证 $succ(x)$ 也满足这一条件。

  根据自然数定义,$succ(x) = x \cup \{x\}$。此时,对于任意 $succ(x)$ 的真子集 $s$:

  如果 $x \notin s$,那么 $s \subsetneqq x$ 根据归纳假设,$\exists z: Nat(z) \land z < x \land s \sim z$,此时也一定有 $z < succ(x)$

  如果 $x \in s$,那么令 $s_0 = s \setminus \{ x \}$,此时 $s_0 \subsetneqq x$。根据归纳假设,$exists z: Nat(z) \land z < x \land s_0 \sim z$,也就是存在双射 $f_0: s_0 \mapsto z$。

  构造一个新的双射:$f: s \mapsto succ(z)$

  $$
    f(n) = \begin{cases}
      f_0(n) & n \neq x \\
      succ(z) & n = x
    \end{cases}
  $$

  因此 $s \sim succ(z)$,而根据 $z < x$,一定有 $succ(z) < succ(x)$。
\end{proof}

\begin{theorem}
  \nameref{aoi} $\implies$ 命题 \ref{pp2}
\end{theorem}

\begin{proof}
  直接构造:$succ: \mathbb{N} \mapsto \mathbb{N} \setminus \{0\}$
\end{proof}

\begin{theorem}
  命题 \ref{pp2} $\implies$ 命题 \ref{pp1}
\end{theorem}

\begin{proof}
  考虑命题 \ref{pp2} 中的集合 A。我们可以通过反证法证明 A 就是我们要找的无穷集合。

  如果 A 是有限集合,那么存在自然数 $x$(在这里,我们不能使用 $\mathbb{N}$)以及双射 $f_1: x \mapsto A$。

  根据命题 \ref{pp2},存在 $B \subsetneqq A$,存在双射 $f_2: A \mapsto B$。

  考虑映射 $g = f_1^{-1} \circ f_2 \circ f_1$。$g(x) \subsetneqq x$。这是一个自然数到自身真子集的双射。根据引理 \ref{nat-ord},这样的双射一定不存在。矛盾。
\end{proof}

\begin{theorem}
  命题 \ref{pp1} $\implies$ \nameref{aoi}
\end{theorem}

\begin{proof}
  令 A 是无穷集合。在接受 AC 的条件下,使用良序定理,可以得到 A 的一个序。

  非正式地,不断取出 A 中的最小元素,我们可以构造出 A 到自然数的一个映射。由于不存在和 A 等势的自然数,因此我们可以得到一个归纳证明,也就是不断得这样取下去。

  将这个映射应用于 \textbf{置换公理模式}(Axiom schema of replacement),我们可以得到一个包含所有自然数的集合的构造。而根据定义这一集合是一个 \nameref{inductive}。
\end{proof}

\end{document}
